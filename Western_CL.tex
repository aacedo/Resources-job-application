%!TEX TS-program = xelatex
%!TEX encoding = UTF-8 Unicode
% Awesome CV LaTeX Template for Cover Letter
%
% This template has been downloaded from:
% https://github.com/posquit0/Awesome-CV
%
% Authors:
% Claud D. Park <posquit0.bj@gmail.com>
% Lars Richter <mail@ayeks.de>
%
% Template license:
% CC BY-SA 4.0 (https://creativecommons.org/licenses/by-sa/4.0/)
%


%-------------------------------------------------------------------------------
% CONFIGURATIONS
%-------------------------------------------------------------------------------
% A4 paper size by default, use 'letterpaper' for US letter
\documentclass[11pt, a4paper]{awesome-cv}

% Configure page margins with geometry
\geometry{left=1.7cm, top=1.3cm, right=1.7cm, bottom=1.8cm, footskip=.5cm}

% Specify the location of the included fonts
\fontdir[1fonts/]

% Color for highlights
% Awesome Colors: awesome-emerald, awesome-skyblue, awesome-red, awesome-pink, awesome-orange
%                 awesome-nephritis, awesome-concrete, awesome-darknight
\colorlet{awesome}{awesome-emerald}
% Uncomment if you would like to specify your own color
% \definecolor{awesome}{HTML}{CA63A8}

% Colors for text
% Uncomment if you would like to specify your own color
% \definecolor{darktext}{HTML}{414141}
% \definecolor{text}{HTML}{333333}
% \definecolor{graytext}{HTML}{5D5D5D}
% \definecolor{lighttext}{HTML}{999999}

% Set false if you don't want to highlight section with awesome color
\setbool{acvSectionColorHighlight}{true}

% If you would like to change the social information separator from a pipe (|) to something else
\renewcommand{\acvHeaderSocialSep}{\quad\textbar\quad}


%-------------------------------------------------------------------------------
%	PERSONAL INFORMATION
%	Comment any of the lines below if they are not required
%-------------------------------------------------------------------------------
% Available options: circle|rectangle,edge/noedge,left/right
%\photo[circle,noedge,left]{profile}
\name{Albert}{Acedo}
\position{Social Geographer{\enskip\cdotp\enskip}GeoInformatics}

\mobile{(+1) 226 898 4081}
\email{albert.acedo@uwaterloo.ca}
%\linkedin{albertacedo}
\skype{albertacedo}
% \reddit{reddit-id}
%\extrainfo{https://bit.ly/2JYnyVs}
% \gitlab{gitlab-id}
% \stackoverflow{SO-id}{SO-name}
\twitter{@acedoalbert}
%\quote{``What we think, we become."}


%-------------------------------------------------------------------------------
%	LETTER INFORMATION
%	All of the below lines must be filled out
%-------------------------------------------------------------------------------
% The company being applied to
\recipient
  {Department of Geography}
  {Social Science Centre\\Western University\\1151 Richmond Street\\London, Ontario, N6A 5C2, Canada}
% The date on the letter, default is the date of compilation
\letterdate{October 15, 2019}
%\letterdate{\today}
% The title of the letter
\lettertitle{Job Application for a Assistant Professor at Western University - Department of Geography (Social Science Centre)}
% How the letter is opened
%\letteropening{Dear Dr. Voogt,}
% How the letter is closed
\letterclosing{Sincerely,}
% Any enclosures with the letter
%\letterenclosure[Attached]{Curriculum Vitae}



%-------------------------------------------------------------------------------
\begin{document}

% Print the header with above personal informations
% Give optional argument to change alignment(C: center, L: left, R: right)
\makecvheader[R]

% Print the footer with 3 arguments(<left>, <center>, <right>)
% Leave any of these blank if they are not needed
\makecvfooter
  {\today}
    {Albert Acedo~~~·~~~Cover Letter}
  {}

% Print the title with above letter informations
\makelettertitle

%-------------------------------------------------------------------------------
%	LETTER CONTENT
%-------------------------------------------------------------------------------
\begin{cvletter}
Dear Dr. Voogt,

I am writing to state my interest in the probationary (tenure-track) appointment in the area of Geographic Information Science (GIScience) and Urban Environments at the rank of Assistant Professor at Western University. I am attracted to this position because my expertise exactly fits with the topics merged (i.e., geinformatics and urban environments) and the perfect timing for a position of these characteristics for my academic goals. I am currently a postdoctoral researcher in the Department of Geography and Environmental Management at the University of Waterloo working with Dr. Peter Johnson. In my research, I study the relationship between social theory, place, and space to operationalize their theoretical knowledge in order to solve problems in the urban domain. I also merge this knowledge with open government, open data and urban policies for new approaches in urban policies and designs. I apply my studies within areas such as planning, smart city developments, environmental psychology, citizen participation, community development, and participatory methodologies. Gathering citizens’ perceptions toward places that they dwell or frequent, I advance on the knowledge of urban systems such as social urban planning and urban human-ecological processes. This research underscores the importance of transferring social-spatial information to the urban domain as an alternative resource for city management practices such as participatory processes.\par


My current postdoctoral project investigates and evaluates the use of platial characteristics, i.e., the place-based geography that is focused on human discourses, social values, and human-space interactions, to describe the relationship between places, citizens, and cities. Such novel approaches will aid the management of urban environments by activating and recognizing the human-space interaction through a social theory lens. Particularly, through merging quantitative and qualitative data, such as urban spatial characteristics, materialities, socio-demographics, citizen cognitive evidence, user-generated data, and place characteristics, we contribute to re-designing cities. This new urban configuration may involve re-imagining boundaries and development strategies, as well as realizing some of the promises of open government, inclusion, and gender equality. Ultimately, the applications of my research may lead to improved government connection to its citizens and underpin individual and collective social, functional, and experiential geographies. One contribution from this research, an analysis of the potential of platial characteristics to define urban areas from the bottom-up perspective, has recently been accepted for publication in Transactions in GIS journal. Another article about the spatialization of degrees of bonding and bridging social capital is under review (GeoJournal). I am also preparing two more publications based on this research; an evaluation of the city spatial characteristics and materialities of urban developments in the development of the sense of place concept, and a more theoretical article about the potential of platial data as applied to different areas of knowledge such as social studies and community-based knowledge. All these academic contributions help to satisfy the pervasive demand for citizen social-spatial information at the city level through a novel perspective based on the so-called platial theory.\par


During my European Joint Doctorate held between Portugal, Spain, and Germany and supervised by full Professor Marco Painho, I evaluated the importance of spatial citizens’ subjectivities, including citizens’ cognitions, feelings, and behaviors toward city places. I formalized and established a robust spatial conceptual framework for understanding sense of place and social capital. This research highlighted (1) the role and opportunities of Geographic Information Science (GISc) and its related tools in defining social-spatial information at the city level, and (2) it empowers all the city stakeholders identifying the pivotal role of understanding the spatial dimension of social concepts in city management practices. I have published five peer-reviewed contributions based on my PhD research (3 years)- three of them in prestigious academic journals and two in relevant conference proceedings.\par

My multidisciplinary background goes from a scientific analytical perspective to related cultural issues. The breadth of my doctoral and postdoctoral experience have helped me contribute across a wide variety of domains of geographic inquiry, including human-environment interaction, urban and regional planning, cultural geography, geospatial technologies, geoinformation, human geography, and platial theory. If I were given the chance to follow my academic career with this position: 1) I would like to build upon visions from the current department themes (e.g., urban sustainability, urban policy, indigenous perspectives, etc.) since I realize that those insights represent a gap in my research, and 2) work together to provide a bridge between platial conceptualizations and other areas of the department. Finally, I would like to incorporate recently developed web-based geographic information systems to support a variety of ongoing and future department projects.\par

In summary, I believe my relevant expertise in geographic information science and social realm, my innovative and interesting research interests, a proven track-record of publishing in high-impact journals, my multidisciplinary background and technical skills make me an ideal candidate to contribute in this offered position.\par

Thank you for your time and consideration. I look forward to hearing from you.\par


\end{cvletter}


%-------------------------------------------------------------------------------
% Print the signature and enclosures with above letter informations
\makeletterclosing

\end{document}
