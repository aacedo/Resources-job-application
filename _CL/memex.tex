%\textbf{\underline{Postdoc position in the ‘SMARTDEST project’}}



Dear Dr. Nisi and Dr. Nunes,

%INTRO

I am writing to state my interest in the positions within the MEMEX project at Interactive Technology Institute (LARSyS). My expertise in social geography, place-related studies, and citizen science in the city context address MEMEX's research goals by dealing with societal challenges through data analysis and innovative social solutions. I am currently a postdoctoral researcher in the Department of Geography and Environmental Management at the University of Waterloo (Canada) working with Dr. Peter Johnson. I apply my studies within areas such as planning, social science, citizen science, community development, and participatory methodologies. I am confident that my academic background and curriculum development skills would be an asset to the project partners to detect and evaluate the transformative force-fields for places and vulnerable communities in assessing their cultural opportunities and resources for social inclusion.


%Research overview

My academic career is mainly connected to the relationship between social theory and its spatial representation which is crucial to better understand the socio-spatial initiatives and practices by citizenship, as well as to deploy innovative designs for sustainability and inclusive processes. I examine the relationship between social theory, place, and space to transfer their socio-theoretical knowledge across city management processes such as planning and decision making. I also merge this knowledge with open government and open data for new approaches in urban policies and designs. Specifically, using theoretical frameworks, spatial and statistical analysis, quantitative and qualitative methods, I gather citizens’ perceptions toward places that they dwell or frequent to advance on the knowledge of urban conflicts and arrangements such as low rates of civic engagement in developed countries. I have published in top-ranked journals in Geographic Information Science (GIScience) such as \textit{ISPRS International Journal of Geo-Information} and multidisciplinary top journals such as \textit{Heliyon}. Currently, an article that analyzes the potential of socio-spatial characteristics to define innovative socio-urban areas from the bottom-up perspective has recently been published in the \textit{Transactions in GIS} journal and another article about the spatialization of degrees of social network-based data in the urban context has been accepted in \textit{GeoJournal}. Overall, this research underscores the importance of transferring social-spatial information to the urban domain using socio-technical systems and GIScience tools as an alternative resource for local communities to reach sustainable, open, and inclusive place-based solutions. It can be remarkably beneficial to assemble that knowledge for cultural heritage research and heritage-related tools to identify, explore, and evaluate the socio-spatial effects at the local and regional levels, for example in the case of Barcelona and Portugal, respectively.\par

%PhD 

During my Marie-Curie European Joint Doctorate (H2020 - GEO-C: enabling open cities) held between Portugal, Spain, and Germany, I evaluated the importance of spatial citizens’ subjectivities, including citizens’ cognitions, feelings, and behaviors toward places under the umbrella of the smart city. I formalized and established a robust spatial conceptual framework for understanding sense of place and social capital. This research highlighted (1) the role and opportunities of GIScience and its related tools in defining social-spatial information at the city level, and (2) it empowers all the city stakeholders identifying the pivotal role of understanding the spatial dimension of social concepts in city management practices. I have published five peer-reviewed contributions based on my PhD research (3 years)- three of them in prestigious academic journals and two in relevant conference proceedings.\par

%POst-doc

My current postdoctoral project investigates and evaluates the use of \emph{platial} characteristics, that is the place-based geography that is focused on human discourses, social values, and human-space interactions, to describe the relationship between places, citizens, and cities. Such novel approach will aid the management of urban environments by activating and recognizing the human-space interaction through a social theory lens. Particularly, through merging quantitative and qualitative data, such as urban spatial characteristics, mobility data, socio-demographics, citizen cognitive evidence, user-generated data, and place characteristics, we contribute to question current city distribution and management when tackling urban management processes or urban conflicts such as the transformation of place identities or community exclusion. This new urban composition may involve re-imagining boundaries and development strategies, as well as realizing some of the promises of open government, social inclusion, and gender equality. Additionally, I research how citizens are involved or engaged in smart city initiatives as well as to evaluate the role of openness in these kind of projects. Ultimately, the applications of my research may lead to improved government connection to its citizens through the articulation of individual and collective social, functional, and experiential geographies and the evaluation of them.\par


%FUTURE 

My future academic goals envision to better accommodate the research that comprises my doctoral and postdoctoral experience to further contribute across a wide variety of geographic domains such as tourism, cultural heritage, mobility studies, gender studies, and platial theory. I am preparing a publication based on the spatial perspective of data collected (~4 thousand records) by the Candian Index of Wellbeing on social isolation through questionnaires, and two collaborative theoretical articles about the potential of a \textit{social datum}, based on platial theory, as the bridge between social theory and community-based knowledge. I am also planning to work on the interrelation between the current hierarchy of administrative boundaries and how those shape citizens' perceptions, functionalities, and behaviors toward places and cultural opportunities. The little literature on transferring new social-spatial practices to city management processes or at least as a resource for triggering alternative geographies to deal with social conflicts such as social cohesion is a gap identified in my research. The positions offered in the project MEMEX gives me the opportunity to tackle this gap applying the breadth of my doctoral and postdoctoral experience toward, for example, new spaces of co-creation that provide inclusive access to tangible and intangible cultural heritage, and providing novel heritage-related tools within the current discourse of urban policies about more cohesive, developed, and empowered communities.\par

\begin{comment}

%TEACHING

My expertise in Geographic Information Science (GIScience), planning, and social theory allow me to teach courses in digital geographies, qualitative and quantitative methods, and GIS, as well as fundamentals in programming language such as SQL and Python. Mixing methods and making associations between cutting-edge tools and theoretical frameworks, my classrooms shift between pragmatic and theoretical perspectives to be able to keep simple and complex overviews needed for the geographical inquiry. Furthermore, I have experience supervising students, helping them to reach their maximum potential by inspiring guiding them through an independent research project.\par

\end{comment}

%SERVICES -

I realized that to transfer the knowledge gained during my academic career needs an effort to mobilize all the stakeholders. Collaboration, participation, and co-construction between partners and stakeholders perform the basic framework to advance in effective research-actions. In this vein, my past experience in a partnership H2020 project made me feel that co-production and association among universities, research centers, and enterprises are the most suitable environment to spread project advances to the broad public. Furthermore, having been involved in different stakeholders such as Canadian NGOs, Lisbon city council, and Barcelona based-ground communities made me deeply consider their expectations regarding these types of projects and how important are workshops, conferences, and services to making information noted, available and accessible to local actors. Indeed, my familiarity with the studied area (i.e., Barcelona and Lisbon) and my connection to actors developing their main research there reinforce my belief about the importance to promote effective and conscient research actions to reach local and regional stakeholders. Based on the project characteristics, MEMEX's team has realized the importance to effectively deploy project outputs and productions and this is a strong reason why I am drawn to this position and its type of work.\par

%SUMMARY

In summary, my innovative research along with my technical skills makes me an ideal candidate to tackle the societal challenges that MEMEX is addressing. Researching at the MEMEX project and being involved with all the partners offer me to connect with prestigious research centers and universities, pursuing my academic interest to gain experience combining different perspectives, cultural backgrounds, and knowledge.\par

Thank you for your time and consideration. I look forward to hearing from you.\par




\begin{comment}
%%%OLD SUMMARY

In summary, I believe my relevant expertise in GIScience and social theory, my innovative and interesting research interests, a proven track record of publishing in high-impact journals, my multidisciplinary background, and technical skills make me an ideal candidate to contribute in this offered position. Researching at the SMARTDEST project and being involved with all the partners offer me to connect with prestigious research centers and universities, pursuing my academic interest to gain experience combining different perspectives, cultural backgrounds, and knowledge.\par



%%%%%%%%%%%%%%%%%%%%%%%%
PROJECT

Objective

The SMARTDEST project tackles the societal challenge of social inclusion and sustainability in European cities by developing innovative solutions in the face of the conflicts and externalities that are emerging as a result of new forms of ‘mobile dwelling’. These encompass the rising cost of living, housing shortages, congestion of public services, the dislocation and marginalisation of low-income workers, and the transformation of place identities; all factors that point at avenues of exclusion of the most vulnerable sectors of resident communities. Faced with this, SMARTDEST’s overarching aim is to contribute towards urban policy agendas which take tourism and its social effects seriously. Its ambition is to fill a knowledge gap about the effects of tourism mobilities on urban inclusion and cohesion, and about the contextual, political and technological factors that determine fundamental variations in such effects; and to explore, design and test the validity of potential innovative pathways to mitigate social exclusion. The project thus includes 4 research packages that respectively (1) analyse tourism mobilities and mobile dwelling as transformative force-fields for places; (2) excavate social exclusion issues and coping practices through the engagement with affected communities in case study cities; (3) develop CityLabs as sites for the design of people-based and place-based solutions (both in the digital and non-digital realm) which demonstrate value for the broad ‘destination ecosystem’ of case study cities, and scale up as innovative systems of governance; (4) transfer the insights gained by the project at local level and extend their impacts through a dialogue with policy entities, concern communities, innovators and scientists throughout the EU policy space. The project is implemented by a consortium of 12 partners from 7 EU countries and 1 associated country, covering a broad range of academic skills; and engages with 8 case study cities.




MAIN POINTS to touch that is related to my topic.

- data important to put in value
- innovation of topic - social innovation - forms of coping and innovative solutions in Barcelona
- Qualitative research on communities


SERVICES

- Set-up of participatory processes in Barcelona
- Collaborative design of solutions through stakeholder engagement in Barcelona
- Design, implementation and participation to dissemination activities documenting the research process and results in Barcelona and across 8 case study cities:



WEAK point

- Tourism
- 


REQUIREMENTS:

- Urban geography, social geography of the city / Urban sociology / Urban economics / Mobilities studies / Sustainable tourism / Basic notions of planning studies, social anthropology, gender studies, social studies of digital technologies are a plus

- Highly skilled in quantitative geo-statistical techniques, use of advanced data management and geo-analysis software (ArcGis, QGis, R)

- Capacity to handle large-scale databases, data transfer protocols

- Skilled in qualitative research with local communities and vulnerable groups or groups at risk of exclusion therein: capacity to handle ethnographic research techniques, participant observations, interviews and focus groups with informants; capacity to respect ethical norms and care in the handling of sensible information

- Knowledge of content analysis, social network-based data scraping techniques and mobility tracking analysis (from GPS, social network data, and other techniques)

- Good communication skills / Leadership capacity
Capacity to connect with policy and social stakeholders (in Barcelona mainly)

- Availability to travel and liaise with other research teams in the consortium



TASKS

The main tasks will be:

Statistical / quantitative research on tourism mobilities across European regions and cities:
Desk research of tourism mobilities and their effects in relation to social exclusion; co-production of position paper
Obtention of data from different data sources (pan EU level) and development of project database and indicators
Measuring and charting of global mobilities and their spatial effects
Exploration of trends of social exclusion across the EU
Obtention of data from different data sources (case study / city level) and statistical analysis of socio-spatial exclusion in 8 'case study cities'
Co-production of scientific report / collection of papers and database on the exclusionary impacts of tourism and other associated mobilities in the EU territory
Statistical / quantitative and qualitative research on tourism mobilities and their social effects in Barcelona:
Exploration of contexts and trends of social exclusion, forms of coping and innovative solutions in Barcelona
Qualitative research on concern communities in relation to tourism in Barcelona
Destination ecosystem mapping and agent-based modelling of case study cities in Barcelona
Co-production of Scientific Report on Local Effects of Tourism and Associated Mobilities as Drivers of Social Exclusion (in Barcelona and comparatively across 8 case study cities) and related papers
Active research: engagement with stakeholders and development of smart solutions to social exclusion produced by tourism mobilities in Barcelona:
Set-up of participatory processes in Barcelona
Collaborative design of solutions through stakeholder engagement in Barcelona
Impact assessment of the adoption of smart solutions in mobility in Barcelona
Analysis of shared value from inclusive solutions for destination ecosystems in Barcelona
Co-production of final scientific report and related papers
Design, implementation and participation to dissemination activities documenting the research process and results in Barcelona and across 8 case study cities:
Policy briefs / Press releases
Website content and social media updates
Communication with - and outreach to - stakeholders for public engagement
Development of content for e-learning tools based on project findings
Planning and organisation scientific events
Presentations of project results at scientific and policy events




NOTES

Who you are?

The cover letter needs to answer these questions.
Why do your fit? Why do you apply?

We need to say our goals, teaching goals, or research goals... You need to present you as unique, and be persuasive... engagen in your letter

A cover letter is a persuasive document that directly connects you to the department, articulating your future contributions.

You need to do research on the department, 

Dont talk a lot about your research if not in the future contributions...

Adjusting your needs, what they need and how you fit there. Based on what you done, and go 

%%%%%%%

TIPS
- 2 pages max. (1 and half pages)
- 11-12 font size - readible... 
- Personalize: 
- Future-oriented and focused
- Provides evidence...

%%%%%%%%

- Introduction

What position are you applying to?

Open up with a story, more narrative it is an option. Be creative. 

Who are you?


Why are you applying (to this job)?

- Major sections: ( Research / Teaching / Service) it depends of the job position, so read very well the position. 

%%%%%%%%%

- Research section

You need to connect with the department, is not one covering that and I am the person for that. 
Mention to collaboration, how your research can connect with other collaborators in the department.
Emphasize the impact of your research, be proud of your research.
Past publish in this, now we are doing this, and in future we will do that... make connections between past present and future

One paragraph about of my research, other about contributions and last one future.
1. research 
2. so far contribution
3. future contributions

%PRACTICAL

First:
Who I am as researcher and how to connect to the department/

Specifically, I focus on this and this and maybe mention your methods.

I publish in these areas or i have under review in these journals. I have been awarded in this and that. I won this money, from this project...

after, talk about your dissertation and how you contribute:
example, my dissertation is that and the impact is this and that...

Future research one or two paragraphs... I am planning to publish that, and also working to another project... like a short proposal of ideas that you have for a second project.

%%%%%%%%%%%%

- Teaching section

1. Overall pedagogical beliefs and practices, as evidenced in past experiences

They want to know how do you teach and what do you teach... Summary of pedagogical, a summary of the teaching statement.

2. Future experiences: what could you can teach and develop?

You are able to share what you can teach and what they are looking for, if you have a contact in the department ask about information how teaching there... 

How you can teach a students from different cultures? What kind of courses you can develop? 

3. Who could you supervise?

You should still say which kind of students you can supervise, articulate that.

%PRACTICAL

first paragraph explain research position and how connects
classes that you tough in the past
and future classes 

How your research informs your teaching, in teaching section viceversa.

%%%%%%%%%%%%%%%

3. Service section

Can include:
- Developing policies, curriculum, admissions, etc

- Student association

- Relevant experiences from off-campus

Say something about how connect your research to your community, you are connected with community

%PRACTICAL



%%%%%

Conclusion

- Summarize the value, you bring to the position, department, and institution.

- Connect to department goals, whether they be research-focused, teaching focused or both!

Check the values of the department and rewrite and put in your conlusion maybe.

%%%%%

Storytelling in cover letters:

One question about the research and after another about methods.

easy sentence... 

My sentence

I work on the relationship between social theory and its spatial representation.

The people who is going to read your cl, maybe it is not familiar with your research, maybe they are not experts. 

Have a sentence prepare. Try to concise your research and appealing to your audience. 

%%%

Research beyond the posting... look for everything list of faculties, instutitions realted, research labs,.. try to indicate haw you are going to collaborate about that.  What the department prioritize? Check the funding they have, and mention that you can help to this resources... how many students have in class as average? Maybe they are replacing someone so go for what you can provide more than him, her.

AVOID THE WORD STUDENT!! FELOWSHIP!!!  I am finishing my post-doc...


%%%. Key considerations

Logical ; if the job is teaching focus on teeaching



%%%%%%%%%%%%%%%%%%%%%%%


RETALES



The adoption of a platial perspective within city management and decision making 

That is, a readily usable tool for practitioners that can enrich several city management processes such as planning


\end{comment}