Finally, I would like to incorporate recently developed web-based geographic information systems to support a variety of ongoing and future department projects.






All these academic contributions help to satisfy the pervasive demand for citizen social-spatial information at the city level through a novel perspective based on the platial theory.




IDEAS



1. Building upon findings of your research. These may relate to findings of your study that you did not anticipate. Moreover, you may suggest future research to address unanswered aspects of your research problem.

 

2. Addressing limitations of your research. Your research will not be free from limitations and these may relate to formulation of research aim and objectives, application of data collection method, sample size, scope of discussions and analysis etc. You can propose future research suggestions that address the limitations of your study.

 

3. Constructing the same research in a new context, location and/or culture. It is most likely that you have addressed your research problem within the settings of specific context, location and/or culture. Accordingly, you can propose future studies that can address the same research problem in a different settings, context, location and/or culture.

 

4. Re-assessing and expanding theory, framework or model you have addressed in your research. Future studies can address the effects of specific event, emergence of a new theory or evidence and/or other recent phenom
