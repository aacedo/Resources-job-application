\textbf{\underline{Job Application for a Assistant Professor at Western University - Department of Geography (Social Science Centre)}}



Dear Dr. Voogt,

%INTRO

I am writing to state my interest in the probationary (tenure-track) appointment in the area of Geographic Information Science (GIScience) and Urban Environments at the rank of Assistant Professor at Western University. My expertise in merging geo-informatics and urban environments accommodates the main topics posted by adopting XX. I am currently a postdoctoral researcher in the Department of Geography and Environmental Management at the University of Waterloo working with Dr. Peter Johnson. I apply my studies within areas such as planning, smart city developments, environmental psychology, citizen participation, community development, and participatory methodologies. I am confident that my academic background and curriculum development skills would be an asset to the department X.

%Research overview

My academic career is mainly connected to the relationship between social theory and its spatial representation that can become/ is crucial / ... I examine the relationship between social theory, place, and space to transfer their theoretical knowledge across city management processes such as planning and decision making. I also merge this knowledge with open government, open data and urban policies for new approaches in urban policies and designs. Using theoretical frameworks, spatial analysis, quantitative, and qualitative methods, I gather citizens’ perceptions toward places that they dwell or frequent to advance on the knowledge of urban systems and arrangements. I published in top-ranked journals in Geographic Information Science such as \textit{Transactions in GIS} and multidisciplinary top journals such as \textit{Heliyon}. Currently, an article that analyze the potential of platial characteristics to define urban areas from the bottom-up perspective, has recently been accepted for publication in the \textit{Transactions in GIS} journal and another article about the spatialization of degrees of bonding and bridging social capital is under review in GeoJournal. Overall, this research underscores the importance of transferring social-spatial information to the urban domain using participatory processes and GIScience tools as an alternative resource for city management practices such as decision-making processes. FALTA FRASE PARA CONECTAR CON LA OFERTA.\par

%PhD

During my European Joint Doctorate held between Portugal, Spain, and Germany, I evaluated the importance of spatial citizens’ subjectivities, including citizens’ cognitions, feelings, and behaviors toward city places. I formalized and established a robust spatial conceptual framework for understanding sense of place and social capital. This research highlighted (1) the role and opportunities of Geographic Information Science (GISc) and its related tools in defining social-spatial information at the city level, and (2) it empowers all the city stakeholders identifying the pivotal role of understanding the spatial dimension of social concepts in city management practices. I have published five peer-reviewed contributions based on my PhD research (3 years)- three of them in prestigious academic journals and two in relevant conference proceedings.\par

%POst-doc

My current postdoctoral project investigates and evaluates the use of \emph{platial} characteristics, that is the place-based geography that is focused on human discourses, social values, and human-space interactions, to describe the relationship between places, citizens, and cities. Such novel approaches will aid the management of urban environments by activating and recognizing the human-space interaction through a social theory lens. Particularly, through merging quantitative and qualitative data, such as urban spatial characteristics, materialities, socio-demographics, citizen cognitive evidence, user-generated data, and place characteristics, we contribute to question current city distribution and management when tackling urban management processes. This new urban composition may involve re-imagining boundaries and development strategies, as well as realizing some of the promises of open government, inclusion, and gender equality. Ultimately, the applications of my research may lead to improved government connection to its citizens through the articulation of individual and collective social, functional, and experiential geographies.\par


%FUTURE

My future academic goals envision to better accommodate the research that comprises my doctoral and postdoctoral experience to further contribute across a wide variety of geographic domains such as urban and regional planning, cultural geography, geospatial technologies, geoinformation, human geography, and platial theory. I am preparing a publication based on the evaluation of city spatial characteristics and materialities of urban developments in the development of human perceptions toward geographical areas, and a theoretical article about the potential of platial theory as the bridge between social theory and community-based knowledge. I am also planning to work on the interrelation between the X and Y as an opportunity to Z in the current research of BB. The poor/inexistence/ connection between X and Y is an academic gap identified in my research. The position offered in the department XX gives me the opportunity to develop the research on XX, working together to provide/address/position ... \par



%TEACHING

My expertise in Geographic Information Science (GIScience), planning, and social theory allow me to teach courses in digital geographies, qualitative and quantitative methods, and GIS, as well as fundamentals in programming language such as SQL and Python. Mixing methods and making associations between cutting-edge tools and theoretical frameworks, my classrooms shift between pragmatic and theoretical perspectives to be able to keep simple and complex overviews needed for the geographical inquiry. Furthermore, I have experience supervising students, helping them to reach their maximum potential by inspiring guiding them through an independent research project.\par


%SERVICES

I realized that to transfer the knowledge gained during my academic career needs an effort to mobilize all the stakeholders. Collaboration, participation, and co-construction between partners perform the basic framework to advance in an effective research-action. Having been involved in different stakeholders such as NGOs, city council, and based-ground communities made me deeply consider their expectations. Workshops, conference, and services to making information noted, available and accessible to local actors is crucial for a real transfer of knowledge. X project has realized that and this is a strong reason for me to apply for this position. 


%SUMMARY

In summary, I believe my relevant expertise in geographic information science and the social realm, my innovative and interesting research interests, a proven track record of publishing in high-impact journals, my multidisciplinary background, and technical skills make me an ideal candidate to contribute in this offered position. Researching and teaching at the department/research group offer me to connect with XX and pursue my academic interest to become a CC using/through/ the GG.\par

Thank you for your time and consideration. I look forward to hearing from you.\par




\begin{comment}
%%%%%%%%%%%%%%%%%%%%%%%%

NOTES

Who you are?

The cover letter needs to answer these questions.
Why do your fit? Why do you apply?

We need to say our goals, teaching goals, or research goals... You need to present you as unique, and be persuasive... engagen in your letter

A cover letter is a persuasive document that directly connects you to the department, articulating your future contributions.

You need to do research on the department, 

Dont talk a lot about your research if not in the future contributions...

Adjusting your needs, what they need and how you fit there. Based on what you done, and go 

%%%%%%%

TIPS
- 2 pages max. (1 and half pages)
- 11-12 font size - readible... 
- Personalize: 
- Future-oriented and focused
- Provides evidence...

%%%%%%%%

- Introduction

What position are you applying to?

Open up with a story, more narrative it is an option. Be creative. 

Who are you?


Why are you applying (to this job)?

- Major sections: ( Research / Teaching / Service) it depends of the job position, so read very well the position. 

%%%%%%%%%

- Research section

You need to connect with the department, is not one covering that and I am the person for that. 
Mention to collaboration, how your research can connect with other collaborators in the department.
Emphasize the impact of your research, be proud of your research.
Past publish in this, now we are doing this, and in future we will do that... make connections between past present and future

One paragraph about of my research, other about contributions and last one future.
1. research 
2. so far contribution
3. future contributions

%PRACTICAL

First:
Who I am as researcher and how to connect to the department/

Specifically, I focus on this and this and maybe mention your methods.

I publish in these areas or i have under review in these journals. I have been awarded in this and that. I won this money, from this project...

after, talk about your dissertation and how you contribute:
example, my dissertation is that and the impact is this and that...

Future research one or two paragraphs... I am planning to publish that, and also working to another project... like a short proposal of ideas that you have for a second project.

%%%%%%%%%%%%

- Teaching section

1. Overall pedagogical beliefs and practices, as evidenced in past experiences

They want to know how do you teach and what do you teach... Summary of pedagogical, a summary of the teaching statement.

2. Future experiences: what could you can teach and develop?

You are able to share what you can teach and what they are looking for, if you have a contact in the department ask about information how teaching there... 

How you can teach a students from different cultures? What kind of courses you can develop? 

3. Who could you supervise?

You should still say which kind of students you can supervise, articulate that.

%PRACTICAL

first paragraph explain research position and how connects
classes that you tough in the past
and future classes 

How your research informs your teaching, in teaching section viceversa.

%%%%%%%%%%%%%%%

3. Service section

Can include:
- Developing policies, curriculum, admissions, etc

- Student association

- Relevant experiences from off-campus

Say something about how connect your research to your community, you are connected with community

%PRACTICAL



%%%%%

Conclusion

- Summarize the value, you bring to the position, department, and institution.

- Connect to department goals, whether they be research-focused, teaching focused or both!

Check the values of the department and rewrite and put in your conlusion maybe.

%%%%%

Storytelling in cover letters:

One question about the research and after another about methods.

easy sentence... 

My sentence

I work on the relationship between social theory and its spatial representation.

The people who is going to read your cl, maybe it is not familiar with your research, maybe they are not experts. 

Have a sentence prepare. Try to concise your research and appealing to your audience. 

%%%

Research beyond the posting... look for everything list of faculties, instutitions realted, research labs,.. try to indicate haw you are going to collaborate about that.  What the department prioritize? Check the funding they have, and mention that you can help to this resources... how many students have in class as average? Maybe they are replacing someone so go for what you can provide more than him, her.

AVOID THE WORD STUDENT!! FELOWSHIP!!!  I am finishing my post-doc...


%%%. Key considerations

Logical ; if the job is teaching focus on teeaching



%%%%%%%%%%%%%%%%%%%%%%%


RETALES



The adoption of a platial perspective within city management and decision making 

That is, a readily usable tool for practitioners that can enrich several city management processes such as planning


\end{comment}