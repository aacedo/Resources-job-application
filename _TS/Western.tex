\textbf{\underline{Teaching Statement}}

\begin{comment}

Beliefs:

1-3

1. Engagement, participation and bla bla
2. Content, 
3. Mentoring

\end{comment}
 

%intro put beliefs like a enumeration 





% Belief 1 - Engagement, effective questions, classroom delivery skills!


% Belief 2 - content: edge-cutting,  group work collaboration!


% Belief 3 - Mentoring, effective questions!!


% Brief statement about your commitment to teaching




As a life-long student, learning for me has always been fun and about making associations: new cutting-edge connections between things I know already, or making associations with new material. As a teacher, I focus on making those connections happen for my students. For me is very important to establish personal strong interchange between me as an instructor and the students in my class. Building a sense of trust, openness and respect is the basis for helping students to combine a good tracking of the class material and create a deeper union with the subject. By exploring the subject all over again, I reconnect with the subject’s essence and main ideas by asking myself questions like "Why is this important?" and "What is the bigger picture here?". These questions guide lectures, class discussions, assignments, and exams. As geography and geographic information science require interdisciplinary approaches, simple and complex overviews, as well as conceptual frameworks are needed. For example, when I taught for international master students in Lisbon as part of my PhD, I incorporated news articles, academic papers, and class material to highlight the relevance of adding transverse approaches to understand the geography of our daily lives. The aim is to plant a help students understand novel approaches to contemporary, old and future issues, highlighting the importance to understand our present from a geographical overview. Big ideas take time to settle, and my goal is to acknowledge the work needed for reasoning and elaborating novel approaches. At the beginning of each course, I tell my students what I expect from them, but I also listen to their ideas, goals and course expectations in a class discussion, so that they can feel comfortable learning in different ways and can help shape their interactive learning experience – even if it partly alters my intention of the course. The main pillars for successful teaching is 1) listening and respecting different opinions; and 2) building trust. My experience as a student and teacher suggest that students who are unclear about expectations often get frustrated and tend to resist learning.\par

I want my students to gain practical life-long learning skills and problem-solving strategies of my students. I focus on creating educational experiences also adapted to millennials. This approach involves using open-source course content that evolves with student feedback, using inquiry in small groups to generate educational experiences, and course experiences from novel open-ended short-term problems that students must solve in either classroom or extracurricular settings using cutting-edge software and tools. However, too often, students perform assignments and collect data with little or no understanding of the geographic principles illustrated by the phenomena. It was not until I had to deal with a new geo-process in QGIS software in graduate school that I found I truly did not understand the logic behind the software tool and the importance to have basic programming skills for the current state of the art in geoinformation systems. I also realized at that point that is crucial to add some additional syllabus through for example massive open online course as alternative resources to have students be able to adjust their daily work instruments rather than just viewing them as ”black boxes”. Consistent with this philosophy, I design laboratory experiments where the focus is the phenomenon being studied and not merely the collection of a large amount of data. For instance, the complexity of urban environments cannot merely be a collection of data from social media but a merge of quantitative and qualitative information that represents as much as possible social, physical, cultural and gender aspects of the urban domain. This can be evaluated in the processing of assignments required along with the hands-on activities as well as writing a report. After completing the experiments and revising it based on my suggestions, students should demonstrate a clear understanding of the different concepts studied and their connections, as well as the geographical foundation that those synergies reveal.\par

Part of being a teacher is being a mentor to students, helping them reach their maximum potential, by being a good role model or guiding them through an independent research project. For example, I am co-supervising an international master student that wants to further develop part of my PhD research. Specifically, he is contributing to better understand the spatial dimension of sense of place in newcomers. Mentoring goes beyond the classroom and provides unique academic experiences that we, as teachers, can not acquire from other academic activities. Working and planning the development of a project requires fostering the development of leadership skills, the correct use of communication tools, and multi-tasking. As a teacher and mentor, I hope to guide and influence students in a positive and subtle ways. Positive attitudes are highly contagious (”what we think, we become”), and being a professor means to foster excitement and inspiration beyond the classroom. In summary, my careful observations of the teaching styles, methods, and philosophies that I experienced throughout my education provide a valuable collection of effective teaching techniques that I draw in subjects such as Geographic Information Science, Social Geography and Data Science. \par
