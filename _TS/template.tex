%\textbf{\underline{Teaching Statement}}

\begin{comment}

Beliefs:

1-3

1. Engagement, participation and bla bla
2. Content, 
3. Mentoring

\end{comment}
 

%intro put beliefs like a enumeration 





% Belief 1 - Engagement, effective questions, classroom delivery skills!


% Belief 2 - content: edge-cutting,  group work collaboration!


% Belief 3 - Mentoring, effective questions!!


% Brief statement about your commitment to teaching


% INTRO

\vspace{5mm}

As a life-long student, learning for me has always been fun and about making associations: relevant connections between old and new material. Through making relationships and learning new teaching methodologies I have recognized three beliefs that will be crucial during my effective teaching career: (1) the connection and communication between teacher and students are vital for powerful and effective teaching; (2) a learning success goes together with innovative practice and high commitment by students; and (3) how to balance the weight of mentoring depending on students' stages make a difference for a prosperous academic track. Those three beliefs describe the main values of my teaching statement, and a synergy among them defines my path to an effective teaching career.\par


% Belief 1 - Engagement, effective questions, classroom delivery skills!



As a teacher, I need to create a learning environment with all students in a given class. I build sense of trust, openness, and respect as the basis for helping students to combine a good tracking of the class material and create a deeper union with the subject. At the beginning of my classes as guest lecturer, I try to recognize the students' expectations and work on the better connection of those to the subject, creating proper engaging environments for effective questioning and adequate learning situations. Students need to feel comfortable by building trust among the class members and improving the sense of collective efficacy. I devote the first part of my classes to discuss what I expect from students, but also listen their ideas, goals and expectations. The result of this interchange needs to be acknowledged as a valuable resource for the subject -  even if it partly alters the main syllabus of the lecture or course. The last part of the class should recap positive and negative experiences during the learning experience in order to put in practice for next teaching experiences.\par



% Belief 2 - Practical & Theoretical Geography - content: edge-cutting,  group work collaboration!


By exploring the subject all over again, I reconnect with the subject’s essence and main ideas by asking myself questions like "Why is this important?" and "What is the bigger picture here?". As geography and geographic information science require interdisciplinary approaches, those questions are fruitful to build needed simple and complex overviews, as well as conceptual frameworks. As geographers and data scientists, we have the advantage to set out, frame and analyze holistic overviews to create value for society. For this purpose, besides a rich fundamental basis, I encourage students to gain practical life-long learning skills and problem-solving strategies through educational experiences involving open cutting-edge content and course content inside or outside university context.  Consistent with this philosophy, laboratory experiments through university resources and massive open online courses allow students to focus on collection, analyses, and visualization of societal challenges mixing the geographic approach with novel spatial and statistical techniques, but, of course, without losing their geographic nature.\par



% Belief 3 - Mentoring - guiding

Part of being a teacher is being a mentor to students, helping them reach their maximum potential, by guiding them through an academic course or an independent research project. For example, I am co-supervising an international master student that wants to further develop part of my PhD research. Mentoring goes beyond the classroom and provides unique academic experiences that can not be acquired from other academic activities. The master student and I have been working and planning the project development trying to help him to foster his leadership skills, the correct use of communication tools, and frequently enhancing his multi-tasking abilities. As a mentor, I hope to guide and influence students in positive and subtle ways. Positive attitudes are highly contagious (”what we think, we become”), and to be a professor means to foster excitement and inspiration beyond the classroom. \par



% Conclusion


In summary, the diversity of teaching styles, methods, and philosophies that I experienced throughout my education, teaching experience and courses provide me a valuable collection of effective teaching techniques that I am able to apply in subjects such as Geographic Information Science, Social Geography and Data Science.\par


\begin{comment}


However, big ideas take time to settle, and our goal as teachers is to acknowledge the work needed for reasoning and elaborating novel approaches.

However, too often, students perform assignments and collect data with little or no understanding of the geographic principles illustrated by the phenomena.

TO-DO







 My experience as a student suggests that an unclear about expectations or they do not feel part of the collective often get frustrated and tend to resist learning. \par




I will design laboratory experiments where the focus is the phenomenon being studied and not merely the collection of a large amount of data.  \par

For instance, the complexity of urban environments cannot merely be a collection of data from social media but a merge of quantitative and qualitative information that represents as much as possible social, physical, cultural and gender aspects of the urban domain.

For example, in my teaching collaborations with international master students in Lisbon, I incorporated news articles, academic papers, and class material to highlight the relevance of adding transverse approaches to understand the geography of our daily live. The aim is to help students understand to realize the importance of old and novel geographic approached to deal with our current societal challenges.



 It was not until I had to deal with a new geo-process in QGIS software in graduate school that I found that I truly did not understand the logic behind the software tool and the importance to have basic programming skills for the current state of the art in geographic information systems. I also realized at that point that is crucial to add some additional syllabus through for example massive open online course as alternative resources to have students be able to adjust their daily work instruments rather than just viewing them as ”black boxes”. 


This can be evaluated in the processing of assignments required along with the hands-on activities as well as writing a report. After completing the experiments and revising it based on my suggestions, students should demonstrate a clear understanding of the different concepts studied and their connections, as well as the geographical foundation that those synergies reveal

\end{comment}
