%!TEX TS-program = xelatex
%!TEX encoding = UTF-8 Unicode
% Awesome CV LaTeX Template for Cover Letter
%
% This template has been downloaded from:
% https://github.com/posquit0/Awesome-CV
%
% Authors:
% Claud D. Park <posquit0.bj@gmail.com>
% Lars Richter <mail@ayeks.de>
%
% Template license:
% CC BY-SA 4.0 (https://creativecommons.org/licenses/by-sa/4.0/)
%


%-------------------------------------------------------------------------------
% CONFIGURATIONS
%-------------------------------------------------------------------------------
% A4 paper size by default, use 'letterpaper' for US letter
\documentclass[11pt, a4paper]{awesome-cv}

\usepackage{comment}

% Configure page margins with geometry
\geometry{left=1.7cm, top=1.3cm, right=1.7cm, bottom=1.8cm, footskip=.5cm}

% Specify the location of the included fonts
\fontdir[1fonts/]

% Color for highlights
% Awesome Colors: awesome-emerald, awesome-skyblue, awesome-red, awesome-pink, awesome-orange
%                 awesome-nephritis, awesome-concrete, awesome-darknight
\colorlet{awesome}{awesome-emerald}
% Uncomment if you would like to specify your own color
% \definecolor{awesome}{HTML}{CA63A8}

% Colors for text
% Uncomment if you would like to specify your own color
% \definecolor{darktext}{HTML}{414141}
% \definecolor{text}{HTML}{333333}
% \definecolor{graytext}{HTML}{5D5D5D}
% \definecolor{lighttext}{HTML}{999999}

% Set false if you don't want to highlight section with awesome color
\setbool{acvSectionColorHighlight}{true}

% If you would like to change the social information separator from a pipe (|) to something else
\renewcommand{\acvHeaderSocialSep}{\quad\textbar\quad}


%-------------------------------------------------------------------------------
%	PERSONAL INFORMATION
%	Comment any of the lines below if they are not required
%-------------------------------------------------------------------------------
% Available options: circle|rectangle,edge/noedge,left/right
%\photo[circle,noedge,left]{profile}
%\name{Albert}{Acedo}
%\position{Social Geographer{\enskip\cdotp\enskip}GeoInformatics}

%\mobile{(+1) 226 898 4081}
%\email{albert.acedo@uwaterloo.ca}
%\linkedin{albertacedo}
%\skype{albertacedo}
% \reddit{reddit-id}
%\extrainfo{https://bit.ly/2JYnyVs}
% \gitlab{gitlab-id}
% \stackoverflow{SO-id}{SO-name}
%\twitter{@acedoalbert}
%\quote{``What we think, we become."}

%Target Information

%%-------------------------------------------------------------------------------
%	LETTER INFORMATION
%	All of the below lines must be filled out
%-------------------------------------------------------------------------------
% The company being applied to
\recipient
  {Ecole nationale supérieure d'architecture et de paysage de Bordeaux}
  {740, cours de la Libération, CS 70109\\Bordeaux, France}
% The date on the letter, default is the date of compilation
\letterdate{October 15, 2019}
%\letterdate{\today}
% The title of the letter
%\lettertitle{Job Application for a Assistant Professor at Western University - Department of Geography (Social Science Centre)}
% How the letter is opened
%\letteropening{Dear Dr. Voogt,}
% How the letter is closed
\letterclosing{Sincerely,}
% Any enclosures with the letter
%\letterenclosure[Attached]{Curriculum Vitae}





%-------------------------------------------------------------------------------
\begin{document}

% Print the header with above personal informations
% Give optional argument to change alignment(C: center, L: left, R: right)
\makecvheader[R]

% Print the footer with 3 arguments(<left>, <center>, <right>)
% Leave any of these blank if they are not needed
%\makecvfooter
%  {\today}
%    {Albert Acedo~~~·~~~Cover Letter}
%  {}

% Print the title with above letter informations
%\makelettertitle

%-------------------------------------------------------------------------------
%	LETTER CONTENT
%-------------------------------------------------------------------------------
\begin{cvletter}

Albert Acedo S\'anchez is conducting a post-doc in the Geography and Environmental Management Department at the University of Waterloo (Ontario, Canada). His main research seeks to understand the potential of Geographic Information Science (GIScience) and its tools in achieving a better understanding of social theory, urban environments, civic engagement, open government, and citizens’ perceptions through the platial theory. He has publications in GIScience and multidisciplinary top journals and is enrolled in projects with Canadian NGO partners (e.g., Open North) and Partnership Grant Geothink.ca. His recent projects have focused on understanding the spatial dimension of social concepts, the potential of platial characteristics to define urban areas from the bottom-up perspective, and the spatialization of degrees of bonding and bridging social capital. Albert is a former member of the GEO-C team (Marie Skłodowska-Curie Actions), where he led the study of civic engagement in smart cities’ environments as his PhD topic. Before that, he took the Erasmus Mundus Master in Geospatial Technologies conducted. He is seeking for new academic opportunities and to exchange ideas about the incorporation and consolidation of social theory in GIScience and vice-versa. 


\end{cvletter}


%-------------------------------------------------------------------------------
% Print the signature and enclosures with above letter informations
%\makeletterclosing

\end{document}
