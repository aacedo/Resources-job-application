%!TEX TS-program = xelatex
%!TEX encoding = UTF-8 Unicode
% Awesome CV LaTeX Template for Cover Letter
%
% This template has been downloaded from:
% https://github.com/posquit0/Awesome-CV
%
% Authors:
% Claud D. Park <posquit0.bj@gmail.com>
% Lars Richter <mail@ayeks.de>
%
% Template license:
% CC BY-SA 4.0 (https://creativecommons.org/licenses/by-sa/4.0/)
%


%-------------------------------------------------------------------------------
% CONFIGURATIONS
%-------------------------------------------------------------------------------
% A4 paper size by default, use 'letterpaper' for US letter
\documentclass[11pt, a4paper]{awesome-cv}

% Configure page margins with geometry
\geometry{left=1.4cm, top=.8cm, right=1.4cm, bottom=1.8cm, footskip=.5cm}

% Specify the location of the included fonts
\fontdir[fonts/]

% Color for highlights
% Awesome Colors: awesome-emerald, awesome-skyblue, awesome-red, awesome-pink, awesome-orange
%                 awesome-nephritis, awesome-concrete, awesome-darknight
\colorlet{awesome}{awesome-emerald}
% Uncomment if you would like to specify your own color
% \definecolor{awesome}{HTML}{CA63A8}

% Colors for text
% Uncomment if you would like to specify your own color
% \definecolor{darktext}{HTML}{414141}
% \definecolor{text}{HTML}{333333}
% \definecolor{graytext}{HTML}{5D5D5D}
% \definecolor{lighttext}{HTML}{999999}

% Set false if you don't want to highlight section with awesome color
\setbool{acvSectionColorHighlight}{true}

% If you would like to change the social information separator from a pipe (|) to something else
\renewcommand{\acvHeaderSocialSep}{\quad\textbar\quad}


%-------------------------------------------------------------------------------
%	PERSONAL INFORMATION
%	Comment any of the lines below if they are not required
%-------------------------------------------------------------------------------
% Available options: circle|rectangle,edge/noedge,left/right
%\photo[circle,noedge,left]{profile}
\name{Luis Fernando}{Santa Guzmán}
\position{Data Scientist{\enskip\cdotp\enskip}Statistician}
%\address{R. Santa Bárbara 30 e 32 R/Ch, 1150-289, Lisbon, Portugal}

\mobile{(+351) 912 960 044}
\email{fernando.santa@novaims.unl.pt}
\linkedin{lfsantag}
\skype{fernando.santa1983}
% \reddit{reddit-id}
% \extrainfo{extra informations}
% \gitlab{gitlab-id}
% \stackoverflow{SO-id}{SO-name}
% \twitter{@twit}
\quote{``What we think, we become."}


%-------------------------------------------------------------------------------
%	LETTER INFORMATION
%	All of the below lines must be filled out
%-------------------------------------------------------------------------------
% The company being applied to
\recipient
  {Karlsruhe Institute of Technology (KIT)}
  {Kaiserstraße 12\\76131\\Karlsruhe\\Germany}
% The date on the letter, default is the date of compilation
\letterdate{\today}
% The title of the letter
\lettertitle{Job Application for a research associate}

% How the letter is opened
\letteropening{Dear Dr. Sebastian Lerch and Prof. Dr. Peter Knippertz}
% How the letter is closed
% How the letter is opened
%\letteropening{prof. dr. ir. (Bernard) Veldkamp}
% How the letter is closed
\letterclosing{Sincerely,}
% Any enclosures with the letter
\letterenclosure[Attached]{Curriculum Vitae}


%-------------------------------------------------------------------------------
\begin{document}

% Print the header with above personal informations
% Give optional argument to change alignment(C: center, L: left, R: right)
\makecvheader[R]

% Print the footer with 3 arguments(<left>, <center>, <right>)
% Leave any of these blank if they are not needed
%\makecvfooter
 % {\today}
  %  {Luis F. Santa G.~~~·~~~Cover Letter}
  %{}

% Print the title with above letter informations
\makelettertitle

%-------------------------------------------------------------------------------
%	LETTER CONTENT
%-------------------------------------------------------------------------------
\begin{cvletter}
%\lettersection{About Me}
I write to show my keen interest in applying to the position as a research associate. I have recently completed the joint doctorate degree in Information Management from the Universidade Nova de Lisboa (Portugal), University of Münster (Germany), and Universitat Jaume I (Spain), as part of the GEO-C project (\url{www.geo-c.eu}), an H2020 ITN Marie Curie Skłodowska action. I have a MSc in Geomatics and two BScs in Statistics and Cadastral and Geodetic Engineering. I consider that my over ten years of academic and professional background experience conducting statistical data analysis, consultancy, research, and teaching have prepared me well to tackle the challenges described as part of the job profile.\par
I am highly motivated by the developed research and education activity at KIT. I also find admirable the scientific infrastructure and the interdisciplinary network that offers to its scientists. Moreover, it is very interesting the possibility of working with scientists from several fields to offer solutions in a broad spectrum of disciplines. My previous experience has been strongly related to define and use statistical modelling for monitoring and predicting processes in several fields as climate, air pollution, public health, ecology, agronomy, transportation, cadastre, geodesy, remote sensing, and urban planning. I have been working with large, noisy, structured and unstructured datasets for several applications. For instance, I studied the distribution of air pollutant (particulate matter 2.5) in Bogota, Colombia, using functional data analysis over the collected data in monitoring stations across the city, the work aimed to understand the spatio-temporal patterns in the distribution of those particles. Also, I developed various projects that aimed to analyse the spatial behaviour of multiple processes that occur in the earth such as subsidence, rainfalls distribution, gravimetric and magnetic fields, among others by using Bayesian methods for predicting and defining optimal spatial sampling designs. I worked with the Colombian Corporation of Agricultural Research and the Colombian Oil Palm Research Centre developing data analysis to study relationships between soil properties and climate variables with crop yield and the distribution of plagues. I have been studying the spatio-temporal behaviour of various diseases such as dengue, malaria, and gastric cancer, through Bayesian regression modelling to identifying several factors associated with the distribution of the number of cases across space and time.\par
Lately, I have been very interested to going in deep in climate modelling, field that I find as an exciting area. Due to the multidisciplinarity of my Ph. D. program I acquired knowledge about ensemble models and skills to deal with climate data operator (CDO) and NetCDF4/HDF5 formats. Currently, I am working with a couple of collegues in developing some alternatives based on spatio-temporal statistical prediction to compare the performance of several downscaling methods of GCM to RCM. Also, I am studying the possibility of using Integrated Nested Laplace Aproximations (INLA) to quantify the uncertainty of climate models.\par
My current work focuses on the use and analysis of spatio-temporal data originated by new and massive data sources such as, for example, social media, sensors, and telecommunication networks that require to rethink software tools, the architecture of systems, and methods of analysis to provide solutions in near real-time. I see myself contributing to the KIT with my experience and knowledge in statistical data analysis. In this sense, I think my profile might provide instruments to develop the project in a very general framework that could cover from systems definition and design to the strategy of data analysis.\par
From my point of view, my academic background is a particular combination that allows me to bring distinct and innovative conceptual elements to fill the position of a research associate. I have profound knowledge in data analysis and, in particular, spatio-temporal prediction using tools from geostatistics, functional data analysis, multivariate time series, extreme value theory, and  Bayesian hierarchical models, among others. I have advanced programming skills which can be seen in my management of R, QGIS, and SQL, which I acquired by catalysing self-learning my interest in always learn new things. On the other hand, I have carried out various positions in different roles, in which I have been able to demonstrate my competence. For instance, I have been the main lecturer at the undergraduate and graduate level in two major universities in Colombia for about eight years, teaching from basic to advanced subjects, around six different classes and between 100 and 140 students per semester. This role has required me to develop group management and communication skills, oral and visual communication tools, and ability to coordinate multiple activities simultaneously. My position as a lecturer has given me the opportunity to work with students in the planning and development of their final bachelor and master projects. Together with my students, we have provided innovative analysis alternatives for multiple and varied phenomena that occur on the earth's surface. Many of these works have been published in research journals and presented at conferences and symposiums. I believe that I am a person who has a robust capability to listen, interpret, and guide others. \par
It is my opinion that becoming part of your research team would be highly rewarding. I believe that I can play a proper role in the activities that the position demands. I am willing to work with commitment, enthusiasm, and dedication to achieve the best results.\par
Thank you for considering my application, and I look forward to future communications with you.\par
Sincerely,\par
{\bf Luis Fernando Santa Guzmán}
\end{cvletter}


%-------------------------------------------------------------------------------
% Print the signature and enclosures with above letter informations
%\makeletterclosing

\end{document}