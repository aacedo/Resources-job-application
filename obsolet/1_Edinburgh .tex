%!TEX TS-program = xelatex
%!TEX encoding = UTF-8 Unicode
% Awesome CV LaTeX Template for Cover Letter
%
% This template has been downloaded from:
% https://github.com/posquit0/Awesome-CV
%
% Authors:
% Claud D. Park <posquit0.bj@gmail.com>
% Lars Richter <mail@ayeks.de>
%
% Template license:
% CC BY-SA 4.0 (https://creativecommons.org/licenses/by-sa/4.0/)
%


%-------------------------------------------------------------------------------
% CONFIGURATIONS
%-------------------------------------------------------------------------------
% A4 paper size by default, use 'letterpaper' for US letter
\documentclass[11pt, a4paper]{awesome-cv}

% Configure page margins with geometry
\geometry{left=1.4cm, top=.8cm, right=1.4cm, bottom=1.8cm, footskip=.5cm}

% Specify the location of the included fonts
\fontdir[fonts/]

% Color for highlights
% Awesome Colors: awesome-emerald, awesome-skyblue, awesome-red, awesome-pink, awesome-orange
%                 awesome-nephritis, awesome-concrete, awesome-darknight
\colorlet{awesome}{awesome-emerald}
% Uncomment if you would like to specify your own color
% \definecolor{awesome}{HTML}{CA63A8}

% Colors for text
% Uncomment if you would like to specify your own color
% \definecolor{darktext}{HTML}{414141}
% \definecolor{text}{HTML}{333333}
% \definecolor{graytext}{HTML}{5D5D5D}
% \definecolor{lighttext}{HTML}{999999}

% Set false if you don't want to highlight section with awesome color
\setbool{acvSectionColorHighlight}{true}

% If you would like to change the social information separator from a pipe (|) to something else
\renewcommand{\acvHeaderSocialSep}{\quad\textbar\quad}


%-------------------------------------------------------------------------------
%	PERSONAL INFORMATION
%	Comment any of the lines below if they are not required
%-------------------------------------------------------------------------------
% Available options: circle|rectangle,edge/noedge,left/right
%\photo[circle,noedge,left]{profile}
\name{Luis F.}{Santa G.}
\position{Data Scientist{\enskip\cdotp\enskip}Statistician}
%\address{R. Santa Bárbara 30 e 32 R/Ch, 1150-289, Lisbon, Portugal}

\mobile{(+351) 912 960 044}
\email{fernando.santa@novaims.unl.pt}
\linkedin{lfsantag}
\skype{fernando.santa1983}
% \reddit{reddit-id}
% \extrainfo{extra informations}
% \gitlab{gitlab-id}
% \stackoverflow{SO-id}{SO-name}
% \twitter{@twit}
%\quote{``What we think, we become."}


%-------------------------------------------------------------------------------
%	LETTER INFORMATION
%	All of the below lines must be filled out
%-------------------------------------------------------------------------------
% The company being applied to
\recipient
  {ExaMotive S.A.}
  {9, avenue du Blues\\4368\\Belvaux\\Luxembourg}
% The date on the letter, default is the date of compilation
\letterdate{\today}
% The title of the letter
\lettertitle{Data Scientist - Smart Mobiliy}

% How the letter is opened
\letteropening{Prof. Ruth King}
% How the letter is closed
\letterclosing{Sincerely,}
% Any enclosures with the letter
\letterenclosure[Attached]{Curriculum Vitae}


%-------------------------------------------------------------------------------
\begin{document}

% Print the header with above personal informations
% Give optional argument to change alignment(C: center, L: left, R: right)
\makecvheader[R]

% Print the footer with 3 arguments(<left>, <center>, <right>)
% Leave any of these blank if they are not needed
\makecvfooter
  {\today}
    {Luis F. Santa G.~~~·~~~Cover Letter}
  {}

% Print the title with above letter informations
\makelettertitle

%-------------------------------------------------------------------------------
%	LETTER CONTENT
%-------------------------------------------------------------------------------
\begin{cvletter}
I write to show my keen interest in applying to the position as a Data Scientist - Smart Mobility at ExaMotive S.A. I have recently completed my PhD in Information Management from the Universidade Nova de Lisboa (Portugal), University of Münster (Germany), and Universitat Jaume I (Spain), as part of the GEO-C project (\url{www.geo-c.eu}), an H2020 ITN Marie Curie Skłodowska action. I have a MSc in Geomatics and two BScs in Statistics and Cadastral and Geodetic Engineering. I consider that my over ten years of academic and professional background experience conducting statistical data analysis, consultancy, research, and teaching have prepared me well to tackle the challenges described as part of the job profile.\par 
I am highly motivated by the type of company and the core business of ExaMotive. My doctoral dissertation was linked with the analysis of transport systems in a smart cities context. Specifically, it was oriented to develop new alternatives for statistical analysis to study urban dynamics. I took advantage that my previous experience has been strongly related to use spatio-temporal statistics for monitoring and predicting processes in several fields as public health, ecology, air pollution, agronomy, climate, cadastre, geodesy, remote sensing, and urban planning. Thus, I implemented a process that involved three stages: (1) data collection from several sources such as social media, road traffic monitoring sensors, and smart cards with entries and exits into an underground railway system, (2) data pre-processing procedures to identify and remove erroneous values, aggregate data, and bring them in the appropriated standards, and (3) data analysis by statistical modelling in R the language and environment for statistical computing.\par
For data analysis, I used statistical modelling since its methods provide elements, parameters of the models, to understand and explain underlying processes that generate the data and can replicate or simulate complex systems. Then, these methods are useful in monitoring urban dynamics more reliably. It considered modelling approaches related to generalised linear models (GLM), functional principal components analysis (FPCA), statistical analysis of spatial point patterns, hierarchical clustering, aberration detection, functional time series, spatio-temporal graph theory, and epidemic data, among others. In most of the cases, it was considered the observations as daily repeated measurements and opted for non-parametric estimation to avoid strong distributional assumptions when it was possible.\par
For social media data, geolocated tweets were downloaded as a proxy for human activity in urban environments. The analysis was divided into two parts. First, by estimating regression models under the scope of the GLMs to explain the number of geolocated tweets per hour in a city as a function of the hour-of-the-day, the day-of-the-week, and autoregressive trends. Second, by clustering hours of the day with similar patterns of spatial arrangement of the places where people interact with their social networks. Additionally, an endemic-epidemic model was estimated assuming a negative binomial distribution for the counts and including seasonal effects as normal or endemic human behaviour and spatio-temporal autoregressive parameters for the epidemic or abnormal situations as crowded events. To study the information coming from road traffic sensors in cities, an approach for characterising, monitoring, and predicting the behaviour of vehicular mobility was developed. First, groups of sensors were identified that exhibit a similar daily distribution in the number of counted vehicles. Second, for each defined group, through implementing techniques of aberration detection a monitoring system of counts of cars to identify out of control counts was built. Finally, predictions of the discrete time series associated with each sensor were made by using Hyndman-Ullah method, the accuracy of the one-step-ahead forecasts was evaluated. Finally, for modelling the spatio-temporal directed graph that represents the flows origin-destination within an underground railway system, daily time series for every pair of stations, with the number of trips every 15 minutes starting in one of them and finishing in the other, were considered. In this case, the method involved a mix between two steps FPCA and hierarchical clustering to summarise daily and weekly behaviours and to describe the activity over the entire graph.\par
Nowadays, my current work focuses on the use and analysis of spatio-temporal data originated by new and massive data sources such as, e.g., social media, sensors, and telecommunication networks that require to rethink software tools, the architecture of systems, and methods of analysis to provide solutions in near real-time. I have been performing different roles including data analyst, researching in interdisciplinary teams, lecturer, and adviser. I see contributing to your lab with my experience. At this point, I am inclined to believe that I am ready to give a jump to a position with more leadership and that demands from me more than of my skills of strategic development and project management. In this sense, I think my profile might provide instruments to develop the project in a very general framework that could cover from definition and design of systems to strategies of the data analysis.\par
From my point of view, my academic background is a particular combination that allows me to bring distinct and innovative conceptual elements to face the position of a data scientist in your company. I have profound knowledge in data analysis and, in particular, spatio-temporal analysis using tools from geostatistics, functional data analysis, multivariate time series, extreme value theory, and Bayesian hierarchical models, among others, which represent a different paradigm within statistics and data mining. I have advanced programming skills which can be seen in my management of R, QGIS, and SQL, which I acquired self-learning following my interest in always learn new things. All of the above has allowed me to carry out various positions in different roles, in which I have been able to demonstrate my competence. For instance, I have been the main lecturer at the undergraduate and graduate level in two major universities in Colombia for about eight years, teaching from basic to advanced subjects, around six different classes and between 100 and 140 students per semester. This role has required me to develop group management and communication skills, oral and visual communication tools, and the ability to coordinate multiple activities simultaneously, among others. My position as a lecturer has given me the opportunity to work with students in the planning and development of their final bachelor and master projects. Together with my students, we have provided innovative analysis alternatives for multiple and varied phenomena that occur on the earth’s surface. Many of these works have been published in research journals and presented at conferences and symposiums. I believe that I am a person who has a robust capability to listen, interpret, and guide others. Moreover, I have worked in interdisciplinary teams in research centres and universities, coordinating, suggesting, and implementing spatio-temporal data analysis.\par
It is my opinion that becoming part of your company would be highly rewarding. I believe that I can play a proper role in the activities that the position demands. I am willing to work with commitment, enthusiasm, and dedication to achieve the best results. In particular, I consider that I might develop the research project integrating to the analysis other data coming from sources such as city sensors and ubiquitous devices, to offer broader solutions to different types of users across several platforms.\par
Thank you for considering my application, and I look forward to future communications with you.\par
Sincerely,\par
{\bf Luis F. Santa G.}
\end{cvletter}


%-------------------------------------------------------------------------------
% Print the signature and enclosures with above letter informations
%\makeletterclosing

\end{document}