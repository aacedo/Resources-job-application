%!TEX TS-program = xelatex
%!TEX encoding = UTF-8 Unicode
% Awesome CV LaTeX Template for Cover Letter
%
% This template has been downloaded from:
% https://github.com/posquit0/Awesome-CV
%
% Authors:
% Claud D. Park <posquit0.bj@gmail.com>
% Lars Richter <mail@ayeks.de>
%
% Template license:
% CC BY-SA 4.0 (https://creativecommons.org/licenses/by-sa/4.0/)
%


%-------------------------------------------------------------------------------
% CONFIGURATIONS
%-------------------------------------------------------------------------------
% A4 paper size by default, use 'letterpaper' for US letter
\documentclass[11pt, a4paper]{awesome-cv}

% Configure page margins with geometry
\geometry{left=1.4cm, top=.8cm, right=1.4cm, bottom=1.8cm, footskip=.5cm}

% Specify the location of the included fonts
\fontdir[fonts/]

% Color for highlights
% Awesome Colors: awesome-emerald, awesome-skyblue, awesome-red, awesome-pink, awesome-orange
%                 awesome-nephritis, awesome-concrete, awesome-darknight
\colorlet{awesome}{awesome-emerald}
% Uncomment if you would like to specify your own color
% \definecolor{awesome}{HTML}{CA63A8}

% Colors for text
% Uncomment if you would like to specify your own color
% \definecolor{darktext}{HTML}{414141}
% \definecolor{text}{HTML}{333333}
% \definecolor{graytext}{HTML}{5D5D5D}
% \definecolor{lighttext}{HTML}{999999}

% Set false if you don't want to highlight section with awesome color
\setbool{acvSectionColorHighlight}{true}

% If you would like to change the social information separator from a pipe (|) to something else
\renewcommand{\acvHeaderSocialSep}{\quad\textbar\quad}


%-------------------------------------------------------------------------------
%	PERSONAL INFORMATION
%	Comment any of the lines below if they are not required
%-------------------------------------------------------------------------------
% Available options: circle|rectangle,edge/noedge,left/right
%\photo[circle,noedge,left]{profile}
\name{Luis F.}{Santa G.}
\position{Data Scientist{\enskip\cdotp\enskip}Statistician}
%\address{R. Santa Bárbara 30 e 32 R/Ch, 1150-289, Lisbon, Portugal}

\mobile{(+351) 912 960 044}
\email{fernando.santa@novaims.unl.pt}
\linkedin{lfsantag}
\skype{fernando.santa1983}
% \reddit{reddit-id}
% \extrainfo{extra informations}
% \gitlab{gitlab-id}
% \stackoverflow{SO-id}{SO-name}
% \twitter{@twit}
%\quote{``What we think, we become."}


%-------------------------------------------------------------------------------
%	LETTER INFORMATION
%	All of the below lines must be filled out
%-------------------------------------------------------------------------------
% The company being applied to
\recipient
  {Department of Geography}
  {University College Cork\\Cork\\Ireland}
% The date on the letter, default is the date of compilation
\letterdate{\today}
% The title of the letter
\lettertitle{Job Application for a Researcher}

% How the letter is opened
\letteropening{Dr. Paul Holloway}
% How the letter is closed
\letterclosing{Sincerely,}
% Any enclosures with the letter
\letterenclosure[Attached]{Curriculum Vitae}


%-------------------------------------------------------------------------------
\begin{document}

% Print the header with above personal informations
% Give optional argument to change alignment(C: center, L: left, R: right)
\makecvheader[R]

% Print the footer with 3 arguments(<left>, <center>, <right>)
% Leave any of these blank if they are not needed
\makecvfooter
  {\today}
    {Luis F. Santa G.~~~·~~~Cover Letter}
  {}

% Print the title with above letter informations
\makelettertitle

%-------------------------------------------------------------------------------
%	LETTER CONTENT
%-------------------------------------------------------------------------------
\begin{cvletter}
I write to show my keen interest in applying to the position as a researcher within the Department of Geography at University College Cork. I have recently completed the joint doctorate degree in Geoinformatics from the Universidade Nova de Lisboa (Portugal), University of Münster (Germany), and Universitat Jaume I (Spain), as part of the GEO-C project (\url{www.geo-c.eu}), an H2020 ITN Marie Curie Skłodowska action. I have a MSc in Geomatics and two BScs in Statistics and Cadastral and Geodetic Engineering. I have a MSc in Geomatics and two BScs in Statistics and Cadastral and Geodetic Engineering. I consider that my over ten years of academic and professional background experience conducting statistical data analysis, geographic information science (GIScience), consultancy, research, and teaching have prepared me well to tackle the challenges described as part of the job profile.\par 
I am highly motivated by the University College Cork  due to its academic and research activity. I also find admirable its working environment and the possibility of professional growth that offers to its employees. Moreover, it is exciting the role that the Department of Geography has been performing and leading in the fields of Geoinformatics and evolving landscape. Moreover, it is very interesting the research of the climate change and its relations with the species. In this sense, I find myself pretty connected with this area since,  my previous experience has been strongly related to the use of spatio-temporal statistics for monitoring and predicting processes in several fields as agronomy, climate, ecology, air pollution, transportation, public health, cadastre, geodesy, remote sensing, and urban planning. I have been working in modelling population dynamics of several species and its relationships with environmental factors. I have been working with large, noisy, structured and unstructured datasets for various applications. For instance, I have been working in modelling population dynamics of several species and its relationships with environmental factors. Correctly, I estimated count regression models to explain the weekly number of catches males of Spodoptera Frugiperda, using sexual pheromones, in two different areas as a function of the food supply in the space. Later, I developed an endemic - epidemic approach based on a branching process to predict the number of caught of Anthonomus Grandis in insect traps in cotton growth fields to establish adequate strategies of pest control. I worked with the Colombian Corporation of Agricultural Research and the Colombian Oil Palm Research Centre developing data analysis to study relationships between soil properties and climate variables with crop yield and the distribution of plagues. During my doctorate, my work was on the study of the urban human dynamics (human activity and human mobility) implementing statistical techniques on dataset coming from social media data, sensors, and telecommunication networks. My current work focuses on the use and analysis of spatio-temporal data originated by new and massive data sources that require to rethink software tools, the architecture of systems, and methods of analysis to provide solutions in near real-time. I see myself contributing to the project with my experience and knowledge in statistical data analysis and GIScience. In this sense, I think my profile might provide instruments to develop the tasks in a very general framework that could cover from systems definition and design to the strategy of data analysis.\par
From my point of view, my academic background is a particular combination that allows me to bring distinct and innovative conceptual elements to fill the position of researcher. I have profound knowledge in data analysis and, in particular, spatio-temporal prediction using tools from geostatistics, functional data analysis, multivariate time series, extreme value theory, and  Bayesian hierarchical models, among others. I have advanced programming skills which can be seen in my management of R, SAS, QGIS, and SQL, which I acquired by catalysing self-learning my interest in always learn new things. On the other hand, I have carried out various positions in different roles, in which I have been able to demonstrate my competence. For instance, I have been the main lecturer at the undergraduate and graduate level in two major universities in Colombia for about eight years, teaching from basic to advanced subjects, around six different classes and between 100 and 140 students per semester. This role has required me to develop group management and communication skills, oral and visual communication tools, and ability to coordinate multiple activities simultaneously. My position as a lecturer has given me the opportunity to work with students in the planning and development of their final bachelor and master projects. Together with my students, we have provided innovative analysis alternatives for multiple and varied phenomena that occur on the earth's surface. Many of these works have been published in research journals and presented at conferences and symposiums. I believe that I am a person who has a robust capability to listen, interpret, and guide others. \par
It is my opinion that becoming part of your research team would be highly rewarding. I believe that I can play a proper role in the activities that the position demands. I am willing to work with commitment, enthusiasm, and dedication to achieve the best results.\par
Thank you for considering my application, and I look forward to future communications with you.\par
Sincerely,\par
{\bf Luis F. Santa G., PhD}
\end{cvletter}


%-------------------------------------------------------------------------------
% Print the signature and enclosures with above letter informations
%\makeletterclosing

\end{document}