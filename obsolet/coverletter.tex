%!TEX TS-program = xelatex
%!TEX encoding = UTF-8 Unicode
% Awesome CV LaTeX Template for Cover Letter
%
% This template has been downloaded from:
% https://github.com/posquit0/Awesome-CV
%
% Authors:
% Claud D. Park <posquit0.bj@gmail.com>
% Lars Richter <mail@ayeks.de>
%
% Template license:
% CC BY-SA 4.0 (https://creativecommons.org/licenses/by-sa/4.0/)
%


%-------------------------------------------------------------------------------
% CONFIGURATIONS
%-------------------------------------------------------------------------------
% A4 paper size by default, use 'letterpaper' for US letter
\documentclass[11pt, a4paper]{awesome-cv}

% Configure page margins with geometry
\geometry{left=1.4cm, top=.8cm, right=1.4cm, bottom=1.8cm, footskip=.5cm}

% Specify the location of the included fonts
\fontdir[fonts/]

% Color for highlights
% Awesome Colors: awesome-emerald, awesome-skyblue, awesome-red, awesome-pink, awesome-orange
%                 awesome-nephritis, awesome-concrete, awesome-darknight
\colorlet{awesome}{awesome-emerald}
% Uncomment if you would like to specify your own color
% \definecolor{awesome}{HTML}{CA63A8}

% Colors for text
% Uncomment if you would like to specify your own color
% \definecolor{darktext}{HTML}{414141}
% \definecolor{text}{HTML}{333333}
% \definecolor{graytext}{HTML}{5D5D5D}
% \definecolor{lighttext}{HTML}{999999}

% Set false if you don't want to highlight section with awesome color
\setbool{acvSectionColorHighlight}{true}

% If you would like to change the social information separator from a pipe (|) to something else
\renewcommand{\acvHeaderSocialSep}{\quad\textbar\quad}


%-------------------------------------------------------------------------------
%	PERSONAL INFORMATION
%	Comment any of the lines below if they are not required
%-------------------------------------------------------------------------------
% Available options: circle|rectangle,edge/noedge,left/right
%\photo[circle,noedge,left]{profile}
\name{Luis Fernando}{Santa Guzmán}
\position{GIS{\enskip\cdotp\enskip}Statistician}
%\address{R. Santa Bárbara 30 e 32 R/Ch, 1150-289, Lisbon, Portugal}

\mobile{(+351) 912 960 044}
\email{fernando.santa@novaims.unl.pt}
\linkedin{lfsantag}
\skype{fernando.santa1983}
% \reddit{reddit-id}
% \extrainfo{extra informations}
% \gitlab{gitlab-id}
% \stackoverflow{SO-id}{SO-name}
% \twitter{@twit}
\quote{``What we think, we become."}


%-------------------------------------------------------------------------------
%	LETTER INFORMATION
%	All of the below lines must be filled out
%-------------------------------------------------------------------------------
% The company being applied to
\recipient
  {The Center for Primary Health Care (CPF)}
  {Faculty of Medicine\\Lund University (LU)\\Malmö\\Sweden}
% The date on the letter, default is the date of compilation
\letterdate{\today}
% The title of the letter
\lettertitle{Job Application for GIS expert/Statistician}

% How the letter is opened
\letteropening{Dear Prof. Dr. Kristina Sundquist}
% How the letter is closed
\letterclosing{Sincerely,}
% Any enclosures with the letter
\letterenclosure[Attached]{Curriculum Vitae}


%-------------------------------------------------------------------------------
\begin{document}

% Print the header with above personal informations
% Give optional argument to change alignment(C: center, L: left, R: right)
\makecvheader[R]

% Print the footer with 3 arguments(<left>, <center>, <right>)
% Leave any of these blank if they are not needed
\makecvfooter
  {\today}
    {Luis F. Santa G.~~~·~~~Cover Letter}
  {}

% Print the title with above letter informations
\makelettertitle

%-------------------------------------------------------------------------------
%	LETTER CONTENT
%-------------------------------------------------------------------------------
\begin{cvletter}
I write to show my keen interest in applying to the position as a GIS expert / Statistician (Reference: PA2019/2163) within the Faculty of Medicine and the Center for Primary Health Care (CPF). I have recently completed the joint doctorate in Geoinformatics from the Universidade Nova de Lisboa (Portugal), University of Münster (Germany), and Universitat Jaume I (Spain), as part of the GEO-C project (\url{www.geo-c.eu}), an H2020 ITN Marie Curie Skłodowska action. I have a MSc in Geomatics and two BScs in Statistics and Cadastral and Geodetic Engineering. I consider that my over ten years of academic and professional background experience conducting research and teaching, as well as supervising graduation projects in Statistics, Geoinformatics, and Geographic Information Systems has prepared me well to tackle the challenges described as part of the job profile.\par
Lund University highly motivates me due to its academic and research activity. I also find admirable its working environment and the possibility of professional growth that offers to its employees. Moreover, it is exciting the role that the Department of Clinical Sciences has been performing and leading on clinical and epidemiological research in areas such as primary health care, population studies, dementia, diabetes, and global health, among others.\par
My previous experience has been strongly related to define and use tools from Geographic Information Systems (GIS) and statistical modelling for monitoring and predicting processes in several fields as public health, air pollution, climate, ecology, agronomy, transportation, cadastre, geodesy, remote sensing, and urban planning. I have been working with large, noisy, structured, and unstructured datasets for various applications. For instance, I conducted the statistical data analysis of a project that aimed to identify and evaluate the effectiveness of a pharmacist-acquired medication history in an emergency department and its effect on reducing the potential adverse drug events. I have also been studying the spatio-temporal behaviour of various diseases such as dengue, malaria, and gastric cancer, through Bayesian regression modelling to establish several factors associated with the distribution of the number of cases across space and time. Furthermore, I have been working in modelling population dynamics of several species and its relationships with environmental factors. Correctly, I estimated count regression models to explain the weekly number of catches males of Spodoptera Frugiperda, using sexual pheromones, in two different areas as a function of the food supply in the space. Later, I developed an endemic - epidemic approach based on a branching process to predict the number of caught of Anthonomus Grandis in insect traps in cotton growth fields to establish adequate strategies of pest control. During my doctorate, my work was on the study of the urban human dynamics (human activity and human mobility) implementing cutting-edge statistical techniques in  open-source GIS platforms, database management systems (DBMS) and statistical software on datasets coming from social media data, sensors, and telecommunication networks. Also, I have developed various projects that aimed to analyse the spatial behaviour of multiple processes that occur in the earth such as subsidence, rainfalls distribution, gravimetric and magnetic fields, among others by using Bayesian methods for predicting and defining optimal spatial sampling designs. I studied the distribution of air pollutant (particulate matter 2.5) in Bogota, Colombia, using functional data analysis over the collected data in monitoring stations across the city, the work aimed to understand the spatio-temporal patterns in the distribution of those particles.\par
My current work focuses on the use and analysis of spatio-temporal data originated by new and massive data sources such as, e.g., social media, sensors, and telecommunication networks that require to rethink software tools, the architecture of systems, and methods of analysis to provide solutions in near real-time. I see contributing to CFP with my experience in researching, consultancy, and  GIS and statistical modelling. In this sense, I think my profile might provide instruments to support your department in a very general framework that could cover from definition and design of systems to strategies of the data analysis.\par
From my point of view, my academic background is a particular combination that allows me to bring distinct and innovative conceptual elements to fill the position of a GIS expert / Statistician. I have profound knowledge in Geospatial tools and data analysis and, in particular, spatio-temporal analysis using tools from databases, spatial analysis, GIS, spatio-temporal statistics, functional data analysis, multivariate time series, extreme value theory, and Bayesian hierarchical models, among others, which represent a different paradigm within statistics and data mining. I have advanced programming skills which can be seen in my management of QGIS, ArcGIS, R, and SQL, which I acquired self-learning following my interest in always learning new things. All of the above has allowed me to carry out various positions in different roles, in which I have been able to demonstrate my competence. For instance, I have been the main lecturer at the undergraduate and graduate level in two major universities in Colombia for about eight years, teaching from basic to advanced subjects, around six different classes and between 100 and 140 students per semester. This role has required me to develop group management and communication skills, oral and visual communication tools, and the ability to coordinate multiple activities simultaneously, among others. My position as a lecturer has allowed me to work with students in the planning and development of their final bachelor and master projects. Together with my students, we have provided innovative analysis alternatives for many and varied phenomena. Concretely, I have promoted the use of GIS and statistics in the analysis and modelling of several fields of knowledge. Many of these works have been published in research journals and presented at conferences and symposiums. I believe that I am a person who has a robust capability to listen, interpret, and guide others. Moreover, I have worked in multidisciplinary teams in research centres and universities, coordinating, suggesting, and implementing data analysis.\par
It is my opinion that becoming part of your research team would be highly rewarding. I believe that I can play a proper role in the activities that the position demands. I am willing to work with commitment, enthusiasm, and dedication to achieve the best results. In particular, I consider that I can implement several methods of geospatial analysis to provide analysis and data integration to offer solutions to the problems that your center face, thus, to give answers to multiple users across several platforms.\par\bigskip
Thank you for considering my application, and I look forward to future communications with you.
\end{cvletter}


%-------------------------------------------------------------------------------
% Print the signature and enclosures with above letter informations
\makeletterclosing

\end{document}
