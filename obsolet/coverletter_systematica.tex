%!TEX TS-program = xelatex
%!TEX encoding = UTF-8 Unicode
% Awesome CV LaTeX Template for Cover Letter
%
% This template has been downloaded from:
% https://github.com/posquit0/Awesome-CV
%
% Authors:
% Claud D. Park <posquit0.bj@gmail.com>
% Lars Richter <mail@ayeks.de>
%
% Template license:
% CC BY-SA 4.0 (https://creativecommons.org/licenses/by-sa/4.0/)
%


%-------------------------------------------------------------------------------
% CONFIGURATIONS
%-------------------------------------------------------------------------------
% A4 paper size by default, use 'letterpaper' for US letter
\documentclass[11pt, a4paper]{awesome-cv}

% Configure page margins with geometry
\geometry{left=1.4cm, top=.8cm, right=1.4cm, bottom=1.8cm, footskip=.5cm}

% Specify the location of the included fonts
\fontdir[fonts/]

% Color for highlights
% Awesome Colors: awesome-emerald, awesome-skyblue, awesome-red, awesome-pink, awesome-orange
%                 awesome-nephritis, awesome-concrete, awesome-darknight
\colorlet{awesome}{awesome-emerald}
% Uncomment if you would like to specify your own color
% \definecolor{awesome}{HTML}{CA63A8}

% Colors for text
% Uncomment if you would like to specify your own color
% \definecolor{darktext}{HTML}{414141}
% \definecolor{text}{HTML}{333333}
% \definecolor{graytext}{HTML}{5D5D5D}
% \definecolor{lighttext}{HTML}{999999}

% Set false if you don't want to highlight section with awesome color
\setbool{acvSectionColorHighlight}{true}

% If you would like to change the social information separator from a pipe (|) to something else
\renewcommand{\acvHeaderSocialSep}{\quad\textbar\quad}


%-------------------------------------------------------------------------------
%	PERSONAL INFORMATION
%	Comment any of the lines below if they are not required
%-------------------------------------------------------------------------------
% Available options: circle|rectangle,edge/noedge,left/right
%\photo[circle,noedge,left]{profile}
\name{Luis F.}{Santa G.}
\position{Data Scientist{\enskip\cdotp\enskip}Statistician}
%\address{R. Santa Bárbara 30 e 32 R/Ch, 1150-289, Lisbon, Portugal}

\mobile{(+351) 912 960 044}
\email{fernando.santa@novaims.unl.pt}
\linkedin{lfsantag}
\skype{fernando.santa1983}
% \reddit{reddit-id}
% \extrainfo{extra informations}
% \gitlab{gitlab-id}
% \stackoverflow{SO-id}{SO-name}
% \twitter{@twit}
%\quote{``What we think, we become."}


%-------------------------------------------------------------------------------
%	LETTER INFORMATION
%	All of the below lines must be filled out
%-------------------------------------------------------------------------------
% The company being applied to
\recipient
  {Luxembourg Institute of Science and Technology (LIST)}
  {5, avenue des Hauts-Fourneaux\\4362\\Esch-Belval\\Luxembourg}
% The date on the letter, default is the date of compilation
\letterdate{\today}
% The title of the letter
\lettertitle{Job application for a post-doctoral position in vehicle data analytics}

% How the letter is opened
%\letteropening{Dear Prof. Dr. Jonathan Bamber and Prof. Dr. Jonathan Rougier}
% How the letter is closed
\letterclosing{Sincerely,}
% Any enclosures with the letter
\letterenclosure[Attached]{Curriculum Vitae}


%-------------------------------------------------------------------------------
\begin{document}

% Print the header with above personal informations
% Give optional argument to change alignment(C: center, L: left, R: right)
\makecvheader[R]

% Print the footer with 3 arguments(<left>, <center>, <right>)
% Leave any of these blank if they are not needed
\makecvfooter
  {\today}
    {Luis F. Santa G.~~~·~~~Cover Letter}
  {}

% Print the title with above letter informations
\makelettertitle

%-------------------------------------------------------------------------------
%	LETTER CONTENT
%-------------------------------------------------------------------------------
\begin{cvletter}

I write to show my keen interest in applying to the position as a Post-doctoral researcher in vehicle data analytics (Job reference:ITIS-2019-018) within the LIST. I have recently completed the joint doctorate degree in Geoinformatics from the Universidade Nova de Lisboa (Portugal), University of Münster (Germany), and Universitat Jaume I (Spain), as part of the GEO-C project (\url{www.geo-c.eu}), an H2020 ITN Marie Curie Skłodowska action. I have a MSc in Geomatics and two BScs in Statistics and Cadastral and Geodetic Engineering. I consider that my over ten years of academic and professional background experience conducting statistical data analysis, consultancy, research, and teaching have prepared me well to tackle the challenges described as part of the job profile.\par 
The LIST highly motivates me due to the research activity developed by its the departments and projects which have been providing, in Luxembourg and European companies, innovative and competitive solutions in areas such as environment, information technology, and materials. I also find admirable the scientific infrastructure and the interdisciplinary network that offers to its scientists. Moreover, it is exciting the possibility of working with researchers from several fields to provide solutions in a broad spectrum of disciplines. In this direction, I feel very connected with the work of The IT for Innovative Services (ITIS) department for developing models and methods for smart systems.\par
My Ph. D. was one of fifteen projects under the GEO-C program (funded by the European Commission under Marie Skłodowska-Curie, International Training Networks and European Joint Doctorates). Its aim was developing methods of data analysis to understand urban dynamics and provide insights into human activity and mobility for urban planning and decision-making processes in the scope of smart cities. Based on the idea that information and communication technologies (ICT) allow collecting data about city environments and human behaviour through using ubiquitous devices such as mobile phones and sensors, I gathered data coming from several sources for a specific city. These sources included: (1) road traffic monitoring sensors, (2) smart cards with entries and exits into a public transport system, and (3) social media. On the other hand, human behaviour presents high spatio-temporal regularity; it is known that people tend to repeat the same activities every 24 hours and frequently visit the same places. However, there are also unexpected situations caused by crowded events, alterations in the traffic due to car accidents, sudden increases in vehicles in the streets, or fail in transport systems.\par
My approach, to model these datasets, was the use of spatio-temporal statistics. Statistical modelling provide elements (parameters of the models) to understand and explain underlying processes generating the data and are able to replicate or simulate complex systems that can be useful in monitoring urban dynamics more reliably. I considered modelling approaches related to generalised linear models (GLM), functional principal components analysis (FPCA), statistical analysis of spatial point patterns, hierarchical clustering, aberration detection, functional time series, spatio-temporal graph theory, and epidemic data, among others. In most of the cases, it was considered the observations as daily repeated measurements and opted for non-parametric estimation to avoid strong distributional assumptions when it was possible.\par
For social media data, geolocated tweets were downloaded as a proxy for human activity in urban environments. The analysis was divided into two parts. First, by estimating regression models under the scope of the GLMs to explain the number of geolocated tweets per hour in a city as a function of the hour-of-the-day, the day-of-the-week, and autoregressive trends. Second, by clustering hours of the day with similar patterns of spatial arrangement of the places where people interact with their social networks. Additionally, I estimated an endemic-epidemic model to explain the number of tweets per area and per hour as a proxy for human activity in cities. This model assumes a negative binomial distribution for the counts and includes seasonal effects as \emph{normal} or \emph{endemic} human behaviour and spatio-temporal autoregressive parameters for \emph{epidemic} or \emph{abnormal} situations as crowed events. To study the information coming from road traffic sensors in cities, an approach for characterising, monitoring, and predicting the behaviour of vehicular mobility was developed. First, groups of sensors were identified that exhibit a similar daily distribution in the number of counted vehicles. Second, for each defined group, through implementing techniques of aberration detection a monitoring system of counts of cars to identify out of control counts was built. Finally, predictions of the discrete time series associated with each sensor were made by using Hyndman-Ullah method, the accuracy of the one-step-ahead forecasts was evaluated. Finally, for modelling the spatio-temporal directed graph that represents the flows origin-destination within an underground railway system, daily time series for every pair of stations, with the number of trips every 15 minutes starting in one of them and finishing in the other, were considered. In this case, the method involved a mix between two steps FPCA and hierarchical clustering to summarise daily and weekly behaviours and to describe the activity over the entire graph.\par
Nowadays, my current work focuses on the use and analysis of spatio-temporal data originated by new and massive data sources such as, e.g., social media, sensors, and telecommunication networks that require to rethink software tools, the architecture of systems, and methods of analysis to provide solutions in near real-time. I have been performing different roles including data analyst, researching in multidisciplinary teams, lecturer, and adviser. I see contributing to your department with my experience. At this point, I am inclined to believe that I am ready to give a jump to a position with more leadership and that demands from me more than of my skills of strategic development and project management. In this sense, I think my profile might provide instruments to develop the project in a very general framework that could cover from definition and design of systems to strategies of the data analysis.\par
From my point of view, my academic background is a particular combination that allows me to bring distinct and innovative conceptual elements to face the position of post-doctoral researcher at your department. I have profound knowledge in data analysis and, in particular, regression modelling, non-parametric statistics, design of experiments, multivariate methods, machine learning, functional data analysis, multivariate time series, extreme value theory, spatio-temporal statistics, and Bayesian hierarchical models, among others. I have advanced programming skills which can be seen in my management of R, QGIS, and SQL, which I acquired self-learning following my interest in always learn new things. All of the above has allowed me to carry out various positions in different roles, in which I have been able to demonstrate my competence. For instance, I have been the main lecturer at the undergraduate and graduate level in two major universities in Colombia for about eight years, teaching from basic to advanced subjects, around six different classes and between 100 and 140 students per semester. This role has required me to develop group management and communication skills, oral and visual communication tools, and the ability to coordinate multiple activities simultaneously, among others. My position as a lecturer has given me the opportunity to work with students in the planning and development of their final bachelor and master projects. Together with my students, we have provided innovative analysis alternatives for multiple and varied phenomena that occur on the earth’s surface. Many of these works have been published in research journals and presented at conferences and symposiums. I believe that I am a person who has a robust capability to listen, interpret, and guide others. Moreover, I have worked in multidisciplinary teams in research centres and universities, coordinating, suggesting, and implementing data analysis.\par
It is my opinion that becoming part of your team would be highly rewarding. I believe that I can play a proper role in the activities that the position demands. I am willing to work with commitment, enthusiasm, and dedication to achieve the best results.\par
Thank you for considering my application, and I look forward to future communications with you.\par
Sincerely,\par
{\bf Luis F. Santa G.}
\end{cvletter}


%-------------------------------------------------------------------------------
% Print the signature and enclosures with above letter informations
%\makeletterclosing

\end{document}